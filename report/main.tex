\documentclass{report}
\usepackage[table,xcdraw]{xcolor}
\usepackage{tikz}
\usepackage{fancyhdr}
\usepackage[utf8]{vietnam}
\usepackage{hyperref} % Hyperlink
\usepackage{parskip}
\usepackage[left=2cm,right=2cm,top=2cm,bottom=2cm]{geometry}
\usetikzlibrary{calc}
\usepackage{titlesec} % Change section font
\usepackage{multirow}
\usepackage{graphicx}
\usepackage{algorithm,algorithmic}
\usepackage{multicol}
\usepackage{setspace}
\usepackage{pdfpages}
\usepackage{lscape}
\usepackage{subfigure}
\usepackage{tabularray}
\usepackage{amsmath}

% ========== [LANGUAGE] ==========
\def\lang{1} % 0 == English, 1 == Vietnamese

\ifnum\lang = 0
    \usepackage[english]{babel}
\fi
% ========== END OF [LANGUAGE] ==========





\begin{document}

% ========== [TITLE PAGE] ==========
\begin{titlepage}

\begin{tikzpicture}[overlay,remember picture]
\draw[line width=4pt]
    ($ (current page.north west) + (1cm,-1cm) $)
    rectangle
    ($ (current page.south east) + (-1cm,1cm) $);
\draw[line width=1.5pt]
    ($ (current page.north west) + (1.2cm,-1.2cm) $)
    rectangle
    ($ (current page.south east) + (-1.2cm,1.2cm) $);
\end{tikzpicture}


\begin{center}
% Upper part of the page
\ifnum\lang = 0
    \textbf{\Large UNIVERSITY OF SCIENCE}\\[0.2cm]
    \textbf{\Large FACULTY OF INFORMATION TECHNOLOGY}\\
\else
    \textbf{\Large TRƯỜNG ĐẠI HỌC KHOA HỌC TỰ NHIÊN}\\
    \textbf{\Large KHOA CÔNG NGHỆ THÔNG TIN}\\
\fi

% University Logo
\begin{figure}[!h]
    \centering
    \includegraphics[width=8cm, height=8cm]{assets/KHTN.png}
\end{figure}

% Title
\rule{\textwidth}{1pt} \\[0.4cm]
{\huge \bfseries BÁO CÁO ĐỒ ÁN - LINEAR REGRESSION}\\[0.4cm]
\textsc{\Large TOÁN ỨNG DỤNG VÀ THỐNG KÊ}
\rule{\textwidth}{1pt} \\[2cm]

% Student name
\begin{center}
    \textbf{\Large HỌ VÀ TÊN: BÙI ĐỖ DUY QUÂN}\\[0.5cm]
    \textbf{\Large MÃ SỐ SINH VIÊN: 21127141}\\[0.5cm]
    \textbf{\Large LỚP: 21CLC02}\\[2cm]
\end{center}

% Advisor name
\begin{center}
    \ifnum\lang = 0
        \textbf{\Large Lecturers: \\[0.2cm]}
    \else
        \textbf{\Large Giảng viên hướng dẫn: \\[0.2cm]}
    \fi
    \Large{Phan Thị Phương Uyên}
\end{center}
\vfill

% Bottom of the page
\ifnum\lang = 0
    \selectlanguage{english} 
\fi
{\large \today}
\end{center}
\end{titlepage}
% ========== END [TITLE PAGE] ==========





% ========== [HEADER AND FOOTER] ==========
\pagestyle{fancy}
\setlength{\headheight}{0.5cm}
\fancyhf{}
\lhead{\textbf{Đồ án 02}}
\rhead{\textbf{Toán ứng dụng và thống kê}}
\ifnum\lang = 0
    \rfoot{Page \thepage}
\else
    \rfoot{Trang \thepage}
\fi
% ========== END [HEADER AND FOOTER] ==========





% ========== [TABLE OF CONTENTS] ==========
\Large
\tableofcontents
\thispagestyle{fancy} % Fix footer and header
\vfill
\pagebreak
% ========== END [TABLE OF CONTENTS] ==========





% ========== [SECTION NUMBERING] ==========
\setcounter{secnumdepth}{10} % Section numbering depth 

\renewcommand\thesection{\arabic{section}} % Section start from 1,2,3...
\renewcommand\thesubsection{\thesection.\arabic{subsection}} % Subsection start from 1,2,3,...
\renewcommand\thesubsubsection{\alph{subsubsection}} 

\titleformat*{\section}{\Large\bfseries}
\titleformat*{\subsection}{\Large\bfseries}
\titleformat*{\subsubsection}{\Large\bfseries}
% ========== END [SECTION NUMBERING] ==========





% ========== [MAIN CONTENT] ==========
\section{YÊU CẦU CỦA ĐỒ ÁN}
\begin{itemize}
    \item Xây dựng mô hình dự đoán \textbf{mức lương} của kỹ sư sử dụng \textbf{mô hình hồì quy tuyến tính (linear regression)} với các yêu cầu sau:
    \begin{enumerate}
        \item Sử dụng 11 đặc trưng gồm: \textbf{Gender, 10percentage, 12percentage, CollegeTier, Degree, collegeGPA, CollegeCityTier, English, Logical, Quant, Domain}
    
        \item Sử dụng 5 đặc trưng tính cách: \textbf{conscientiousness, agreeableness, extraversion, nueroticism, openess\_to\_experience} và sử dụng phương pháp \textbf{k-fold cross validation} tìm ra đặc trưng tốt nhất trong các đặc trưng tính cách.
        
         \item Sử dụng 3 đặc trưng: \textbf{English, Logical, Quant} và sử dụng phương pháp \textbf{k-fold cross validation} tìm ra đặc trưng tốt nhất.
 
        \item Sinh viên tự xây dựng các mô hình (tối thiểu 3) và tìm mô hình cho kết quả tốt nhất qua phương pháp \textbf{k-fold cross validation}.
    \end{enumerate}

    \item Các thư viện được cho trước: \textbf{Numpy}, \textbf{pandas}.
\end{itemize}


\section{BỘ DỮ LIỆU}
\begin{itemize}
    \item  Bộ dữ liệu \textbf{\href{https://www.kaggle.com/datasets/manishkc06/engineering-graduate-salary-prediction}{Engineering Graduate Salary}} gồm 2998 dòng và 34 cột. Sau quá trình tiền xử lý là loại bỏ các cột có \textbf{giá trị chuỗi} và \textbf{giá trị liên quan đến định danh và năm} thì còn lại 2998 dòng và 24 cột như sau:
    \begin{itemize}
        \item Giá trị mục tiêu (y): \textbf{Salary}
        \item 23 đặc trưng giải thích (X) gồm: Gender, 10percentage, 12percentage, CollegeTier, Degree, collegeGPA, CollegeCityTier, English, Logical, Quant, Domain, ComputerProgramming, ElectronicsAndSemicon, ComputerScience, MechanicalEngg, ElectricalEngg, TelecomEngg, CivilEngg, conscientiousness, agreeableness, extraversion, nueroticism, openess\_to\_experience.
    \end{itemize}
    \item Sinh viên đã được cung cấp 2 bộ dữ liệu: \textbf{train.csv} và \textbf{test.csv}. Bộ dữ liệu \textbf{train.csv} gồm 2248 mẫu để huấn luyện mô hình , và bộ dữ liệu \textbf{test.csv} gồm 750 mẫu để kiểm tra mô hình.
\end{itemize}

\section{CÁC THƯ VIỆN SỬ DỤNG THÊM}
Ngoài việc sử dụng 2 thư viện được cung cấp là \textbf{Numpy} và \textbf{pandas}, sinh viên còn sử dụng thêm thư viện \textbf{sklearn} và sử dụng module \textbf{feature\_selection} của thư viện này. Trong module này sẽ sử dụng 3 lớp:
\begin{itemize}
    \item \textbf{VarianceThreshold}: Loại bỏ các đặc trưng có \textbf{phương sai} nhỏ hơn ngưỡng được đặt trước.
    \item \textbf{mutual\_info\_regression}: Tính độ tương quan giữa các đặc trưng và giá trị mục tiêu.
    \item \textbf{SelectPercentile}: Chọn ra nhóm các đặc trưng có độ tương quan cao nhất với giá trị mục tiêu.
\end{itemize}

\section{KIẾN THỨC NGHIÊN CỨU}
\subsection{Mô hình hồi quy tuyến tính}
\section{TÀI LIỆU THAM KHẢO}
\begin{itemize}
    \item Cô Phan Thị Phương Uyên.
    \item 
    \href{https://ie.nitk.ac.in/blog/2020/01/19/algorithms-for-adjusting-brightness-and-contrast-of-an-image/}{Công thức thay đổi độ sáng và độ tương phản}.

    \item
    \href{https://www.dfstudios.co.uk/articles/programming/image-programming-algorithms/image-processing-algorithms-part-5-contrast-adjustment/}{Thay đổi độ tương phản}.

    \item 
    \href{https://www.baeldung.com/cs/convert-rgb-to-grayscale}{Công thức chuyển thành hình xám}.

    \item 
    \href{https://dyclassroom.com/image-processing-project/how-to-convert-a-color-image-into-sepia-image}{Công thức chuyển thànhh hình màu sepia}.

    \item \href{https://en.wikipedia.org/wiki/Kernel_(image_processing)}{Công thức và ma trận các kernel để làm mờ và rõ nét ảnh}.

    \item 
    \href{https://www.youtube.com/watch?v=4Eh0y3LHTNU&t=658s}{Triển khai thuật toán cho làm mờ ảnh}.
    \item \href{https://www.maa.org/external_archive/joma/Volume8/Kalman/General.html}{Phương trình ellip}.

    \item \href{https://math.stackexchange.com/questions/2003517/how-to-calculate-width-and-height-of-a-45-rotated-ellipse-bounded-by-a-square}{Công thức của ellip khi xoay 45 độ}.

    \item \href{https://numpy.org/doc/stable/user/index.html#user}{Tài liệu các hàm trong thư viện numpy}.

    \item \href{https://matplotlib.org/stable/api/_as_gen/matplotlib.pyplot.imshow.html}{Thư viện matplotlib hỗ trợ xuất ảnh và lưu ảnh}.

    \item \href{https://pillow.readthedocs.io/en/stable/reference/Image.html}{Thư viện Pillow hỗ trợ đọc ảnh}.
    
\end{itemize}
% ========== END [MAIN CONTENT] ==========
\end{document}